\subsection{Plots}
\begin{itemize*}
	\item for testing and debugging framework functionality there are some prepared plot functions residing in namespace \code{cdf.plot}
	\item all of these functions do not show any plots, instead they write images to disk
\end{itemize*}

\begin{tabular}{l}
	\hline
	\code{\textbf{plot.sync}( run, offs, plotfile )}\\
	\hline
\end{tabular}
\begin{itemize*}
	\item this function plots sync marker offsets to \code{plotfile}
	\item experimental data are given in input \code{run} of type \code{hRun}
	\item input vector \code{offs} needs to be specified as returned by \code{sync} function
\end{itemize*}
\medskip

\begin{tabular}{l}
	\hline
	\code{\textbf{plot.trial\_range}( run, cfg, trial, range, rzp, plotfile )}\\
	\hline
\end{tabular}
\begin{itemize*}
	\item this function plots a trial range to \code{plotfile}
	\item the plot includes two-channel audio data, full bandwidth spectrogram, response ranges and landmarks
	\item experimental data are given by input \code{run} of type \code{hRun} using \code{trial} of type \code{hTrial}
	\item plot range can be adjusted by input \code{range} with zero point \code{rzp}
	\item input configuration \code{cfg} of type \code{hConfig} is used for spectrogram generation (\code{.sta\_*})
\end{itemize*}
\medskip

\begin{tabular}{l}
	\hline
	\code{\textbf{plot.extract}( run, detected, labeled, plotfile )}\\
	\hline
\end{tabular}
\begin{itemize*}
	\item this function plots response extraction accuracies to \code{plotfile}
	\item the plot includes range start and stop deltas and range overlap, pointless without any annotation data
	\item experimental data are given by input \code{run} of type \code{hRun} and matrices \code{detected} and \code{labeled} holding detected and annotated response ranges
\end{itemize*}
\medskip

\begin{tabular}{l}
	\hline
	\code{\textbf{plot.trial\_extract}( run, cfg, trial, plotfile )}\\
	\hline
\end{tabular}
\begin{itemize*}
	\item this function plots extraction internals to \code{plotfile}
	\item the plot includes response audio data, denoised total power and voice activity
	\item experimental data are given by input \code{run} of type \code{hRun} using \code{trial} of type \code{hTrial}
	\item actually this function re-extracts response range from a single trial using input configuration \code{cfg} of type \code{hConfig}
\end{itemize*}
\medskip

\begin{tabular}{l}
	\hline
	\code{\textbf{plot.landmark}( run, detected, labeled, plotfile )}\\
	\hline
\end{tabular}
\begin{itemize*}
	\item this function plots landmark detection accuracies to \code{plotfile}
	\item the plot includes burst-onset, voice-onset and voice-release deltas, pointless without any annotation data
	\item experimental data are given by input \code{run} of type \code{hType} and matrices \code{detected} and \code{labeled} holding detected and annotated landmarks
\end{itemize*}
\medskip

\begin{tabular}{l}
	\hline
	\code{\textbf{plot.trial\_glottis}( run, cfg, trial, plotfile )}\\
	\hline
\end{tabular}
\begin{itemize*}
	\item this function plots glottis landmark detection internals to \code{plotfile}
	\item the plot includes response subband spectrogram, denoised maximum power and rate-of-rises with peaks
	\item experimental data are given by input \code{run} of type \code{hRun} using \code{trial} of type \code{hTrial}
	\item actually this function re-detects glottis landmarks in a single trial using input configuration \code{cfg} of type \code{hConfig}
\end{itemize*}
\medskip

\begin{tabular}{l}
	\hline
	\code{\textbf{plot.trial\_burst}( run, cfg, trial, plotfile )}\\
	\hline
\end{tabular}
\begin{itemize*}
	\item this function plots burst landmark detection internals to \code{plotfile}
	\item the plot includes response audio data and plosion index
	\item experimental data are given by input \code{run} of type \code{hRun} using \code{trial} of type \code{hTrial}
	\item actually this function re-detects burst landmark in a single trial using input configuration \code{cfg} of type \code{hConfig}
\end{itemize*}

\begin{tabular}{l}
	\hline
	\code{\textbf{plot.timing}( run, detected, labeled, plotfile )}\\
	\hline
\end{tabular}
\begin{itemize*}
	\item this function plots landmark timings to \code{plotfile}
	\item the plot includes voice-onset time, vowel and syllable lengths
	\item experimental data are given by input \code{run} of type \code{hType} and matrices \code{detected} and \code{labeled} holding detected and optional annotated landmarks
\end{itemize*}
\medskip

\begin{tabular}{l}
	\hline
	\code{\textbf{plot.babbling}( pows, freqs, plotfile )}\\
	\hline
\end{tabular}
\begin{itemize*}
	\item this function plots babbling power spectrum to \code{plotfile}
	\item spectrum data are specified by input vectors \code{pows} and \code{freqs} as returned by \code{babbling}
\end{itemize*}
\medskip

