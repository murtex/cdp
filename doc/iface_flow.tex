\subsection{Workflow}
\begin{itemize*}
	\item the following passes are proposed in order of processing
	\item passes may be processed on single or multi-run basis, but for each single run the given order has to be followed
	\item it is suitable to write data to disk after each pass to enable pass-wise testing and debugging
\end{itemize*}

\paragraph{Raw conversion}
\begin{itemize*}
	\item this pass is totally application-specific and depends on experimental data format
	\item at the end of this pass application should provide a valid \code{hRun} structure containing proper \code{hTrial} structures
\end{itemize*}

\paragraph{Syncing}
\begin{itemize*}
	\item as there is a still not understood asynchrony between trigger and recording devices timings need to be synced
	\item in fact this pass is optional and not needed for short recordings, for large recordings it is crucial
	\item upon completion all trial timings are adjusted to fit with sync marker positions
\end{itemize*}

\paragraph{Response extraction}
\begin{itemize*}
	\item the first pass which actually involves detection techniques is the coarse extraction of subject's speech from recording
	\item this is the slowest pass as it works on complete trial audio data and full bandwidth spectrum
	\item after processing trial response ranges (\code{hTrial.detected.range}) are set properly
	\item its primary goal is data reduction for further processing
\end{itemize*}

\paragraph{Landmark detection}
\begin{itemize*}
	\item this pass is internally split into glottis activity and burst detection
	\item it processes rather fast since it works only on partial audio data and a spectral subband
	\item after execution response landmarks (\code{hTrial.detected.bo}, \code{.vo} and \code{.vr}) are set
\end{itemize*}

\paragraph{Babbling spectrum}
\begin{itemize*}
	\item this pass can be performed on both detected and labeled data
	\item it estimates the average power spectrum of response speech parts
\end{itemize*}

